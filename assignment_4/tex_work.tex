\documentclass[11pt, oneside]{article}   	% use "amsart" instead of "article" for AMSLaTeX format
\usepackage{geometry}                		% See geometry.pdf to learn the layout options. There are lots.
\geometry{letterpaper}                   		% ... or a4paper or a5paper or ... 
%\geometry{landscape}                		% Activate for rotated page geometry
%\usepackage[parfill]{parskip}    		% Activate to begin paragraphs with an empty line rather than an indent
\usepackage{graphicx}				% Use pdf, png, jpg, or eps§ with pdflatex; use eps in DVI mode
								% TeX will automatically convert eps --> pdf in pdflatex
								
\usepackage{amssymb}
\usepackage{amsmath}
%SetFonts

%SetFonts


\title{TeX work for Assignment 4}
\author{Michael}
%\date{}							% Activate to display a given date or no date

\begin{document}
\maketitle

\section{}

Let $\beta = $ fraction of workers going from New York to Chicago.

Let $\gamma= $ fraction of workers going from Chicago to New York.

Then the laws of motion for $N_{t+1}$, $C_{t+1}$, and $P_{t+1}$ are:

$N_{t+1} = (1-\beta)N_t + \gamma C_t$

$C_{t+1} = (1-\gamma)C_t + \beta N_t$

$P_{t+1} = P_t$ since there are no births or deaths in this particular Lake Model.

\section{}

The A matrix is as follows:

$\begin{bmatrix}
1-\beta & \gamma \\
\beta & 1-\gamma
\end{bmatrix}$

\section{}
Because there are no births nor are there deaths, then the growth rate, $g$ = 0. 
Thus normally in this case, $\hat{A} = \frac{1}{1+g}*A$ but since $g=0$ then $\hat{A}=A$

\section{}
To find the steady state, we must find the eigenvector [n,c] for A given the $\lambda = 1$ to satisfy the equation: 

$\hat{A}\left[ \begin{array}{c} n \\ c \end{array} \right] = \lambda \left[ \begin{array}{c} n \\ c \end{array} \right]$ where $\lambda = 1$. 

When $\lambda = 1$ we have the following equation for the nullspace of $\hat{A}$. 

$\begin{bmatrix} -\beta & \gamma \\ \beta & -\gamma \end{bmatrix} \times 
\left[ \begin{array}{c} n \\ c \end{array} \right] = 
\left[ \begin{array}{c} 0 \\ 0 \end{array} \right]$


$\begin{bmatrix} -\beta & \gamma \\ 0 & 0 \end{bmatrix} \times 
\left[ \begin{array}{c} n \\ c \end{array} \right] = 
\left[ \begin{array}{c} 0 \\ 0 \end{array} \right]$


$\begin{bmatrix} 1 & \frac{-\gamma}{\beta} \\ 0 & 0 \end{bmatrix} \times 
\left[ \begin{array}{c} n \\ c \end{array} \right] = 
\left[ \begin{array}{c} 0 \\ 0 \end{array} \right]$


Therefore, $n-\frac{\gamma}{\beta}c = 0$, which implies that the corresponding eigenvector to 
$\lambda = 1$ is c$\left[ \begin{array}{c} \frac{\gamma}{\beta} \\ 1 \end{array} \right]$ where $c \in R$ (reals). Thus given a certain $\gamma$ and $\beta$ the eigenvector should be adjusted accordingly. 

\section{}
The probability that a worker is in Chicago at time $t$ and ends up in New York at time $t+3$ is determined by the transition matrix:

P = $\begin{bmatrix} 1-\beta & \gamma \\ \beta & 1-\gamma \end{bmatrix}$, where $\beta$ represents the transition probability from New York to Chicago, and $\gamma$ represents the transition probability from Chicago to New York.

The four paths by which a worker can start in Chicago and end up in New York from time $t$ to time $t+3$ are as follows:

(1) $C\rightarrow C\rightarrow C \rightarrow N$, which corresponds to the probability $(1-\gamma)^2\gamma$

(2) $C\rightarrow C\rightarrow N \rightarrow N$, which corresponds to the probability $(1-\gamma)(\gamma)(1-\beta)$

(3) $C \rightarrow N \rightarrow C \rightarrow N$, which corresponds to the probability $(\gamma)(\beta)(\gamma)$

(4) $C \rightarrow N \rightarrow N \rightarrow N$, which corresponds to the probability $(\gamma)(1-\beta)^2$




\end{document}  